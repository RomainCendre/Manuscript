%% ------------------
%% Helpers when writing (display the TODO notes and TODO list)
%% ------------------

\usepackage{todonotes}[french]
\usepackage{etoolbox}

\newtoggle{todolist}
\togglefalse{todolist}

\newcommand{\phdseetodo}{\toggletrue{todolist}}
\newcommand{\phdhidetodo}{\togglefalse{todolist}}
\newcommand{\phdtodotoc}{\iftoggle{todolist}{\clearpage\listoftodos}{}}
\newcommand{\phdtodomargin}[2][green]{\iftoggle{todolist}{\todo[color=#1!40]{#2}}{}}
\newcommand{\phdtodo}[2][green]{\iftoggle{todolist}{\todo[inline, color=#1!40]{#2}}{}}
\newcommand{\phdmissingfig}[2]{\missingfigure{#1}}


\usepackage{subcaption}
\usepackage{wrapfig}
\usepackage{pdflscape}
\usepackage{multirow, makecell}
\usepackage{array,booktabs}

%% ----------------------------------------------------------------
\usepackage[xindy,acronym,toc,shortcuts]{glossaries}
\usepackage[xindy]{imakeidx}

%% ----------------------------------------------------------------
%% --- Configure hyperlinks, minitoc, epigraph
%% ----------------------------------------------------------------
\usepackage{xcolor}

\usepackage{minitoc}
\usepackage{epigraph}

% Include the listings-package for code listings
\usepackage{algpseudocode}
\usepackage{algorithm}

\usepackage[T1]{fontenc}
\usepackage[utf8]{inputenc}
\usepackage{csquotes}

\usepackage{amsmath}
\usepackage{bibentry}
\usepackage{cleveref}
\usepackage[inline]{enumitem}
\usepackage{siunitx}
\usepackage{subcaption}
% \usepackage{fancyhdr}
% \pagestyle{fancy}

% \renewcommand{\headrulewidth}{0pt} 
% \fancyhead[LE, LO] {Page \thepage}
% \fancyhead[RE, RO] {Question \the\QuestionNumber}

%%%%%%%%%%%%%%% Parameters
% Custom hyperlinks
\hypersetup{
  colorlinks,
  linkcolor={red!80!black},
  citecolor={blue!70!black},
  urlcolor={blue!50!black},
}

% Length of epigraph
\setlength{\epigraphwidth}{0.7\textwidth}

% Depth of minitoc of 2
\setcounter{minitocdepth}{2}

% Custom list
\newlist{inlinerate}{enumerate*}{1}
\setlist[inlinerate]{label=(\roman*)}

% Custom unit and range
\sisetup{range-phrase=--}
\DeclareSIUnit\px{px}

\nobibliography*

% Custom chapter
\newcommand\unchapter[1]{%
  \chapter*{#1}%
  \chaptermark{#1} 
  \addcontentsline{toc}{chapter}{#1}
  \mtcaddchapter}
  
% Custom title for mini toc contents
\renewcommand{\mtctitle}{Sommaire}

% Custom caption
\DeclareCaptionLabelSeparator{colon}{ : }

% - Use column specifier R/C/L{size} when declaring array to fix column size
\newcolumntype{R}[1]{>{\raggedleft\let\newline\\\arraybackslash\hspace{0pt}}m{#1}}
\newcolumntype{C}[1]{>{\centering\let\newline\\\arraybackslash\hspace{0pt}}m{#1}}
\newcolumntype{L}[1]{>{\raggedright\let\newline\\\arraybackslash\hspace{0pt}}m{#1}}
  
%%%%%%%%%%%%%%% Special macros
\makeglossaries
\makeindex

\makeatletter
\newcommand*{\glsplainhyperlink}[2]{%
  \colorlet{currenttext}{.}% store current text color
  \colorlet{currentlink}{\@linkcolor}% store current link color
  \hypersetup{linkcolor={red!40!black}}% set link color to text color
  \hyperlink{#1}{#2}% create link
  \hypersetup{linkcolor=currentlink}% reset link color to original
}
\let\@glslink\glsplainhyperlink
\makeatother

% -- adding keys for multilingual acronyms
% -- french name and french acronym
\glsaddkey{frshort}
{}% default value
{\glsentryfrshort}
{\Glsentryfrshort}
{\acrfrshort}
{\Acrfrshort}
{\ACRfrshort}

\glsaddkey{frlong}
{}
{\glsentryfrlong}
{\Glsentryfrlong}
{\acrfrlong}
{\Acrfrlong}
{\ACRfrlong}

% -- test if new keys are present in entry
\newcommand*{\glsifhasfrshort}[3]{%
  \ifcsempty{glo@#1@frshort}{#3}{#2}%
}

\newcommand*{\glsifhasfrlong}[3]{%
  \ifcsempty{glo@#1@frlong}{#3}{#2}%
}

\newcommand*{\glsifhasfr}[3]{%
  \glsifhasfrshort{#1}{#2}{\glsifhasfrlong{#1}{#2}{#3}}
}

\newcommand*{\acrshortfr}[1]{%
  \glsifhasfrshort{#1}{\acrfrshort{#1}}{\acrshort{#1}}%
}

\newcommand*{\acrlongfr}[1]{%
  \glsifhasfrlong{#1}{\acrfrlong{#1}}{\acrlong{#1}}%
}

% -- define new acronym style
\newacronymstyle{fracr}
{%
  \GlsUseAcrEntryDispStyle{long-short}%
}%
{% base definition on 'long-short'
  % make some modifications for first use display
  % Singular, no case change:
  \GlsUseAcrStyleDefs{long-short}%
  \renewcommand*{\genacrfullformat}[2]{%
    \glsifhasfr{##1}%
    {%
      \glsentrylong{##1}##2,%
      \space\firstacronymfont{\glsentryshort{##1}}\space%
      (\textit{fr : \glsifhasfrlong{##1}%
        {\glsentryfrlong{##1}##2\glsifhasfrshort{##1}{, \space}{}}{}%
        \glsifhasfrshort{##1}{\glsentryfrshort{##1}}{}})%
    }%
    {%
      \glsentrylong{##1}##2,
      \space(\firstacronymfont{\glsentryshort{##1}})%
    }%
  }%
  % Singular, first letter upper case
  \renewcommand*{\Genacrfullformat}[2]{%
    \glsifhasfr{##1}%
    {%
      \Glsentrylong{##1}##2,
      \space\firstacronymfont{\glsentryshort{##1}}\space
      (\textit{fr : \glsifhasfrlong{##1}%
        {\glsentryfrlong{##1}##2\glsifhasfrshort{##1}{, \space}{}}{}%
        \glsifhasfrshort{##1}{\glsentryfrshort{##1}}{}})%
    }%
    {%
      \Glsentrylong{##1}##2,
      \space(\firstacronymfont{\glsentryshort{##1}})%
    }%
  }%
  % Plural, no case change:
  \renewcommand*{\genplacrfullformat}[2]{%
    \glsifhasfr{##1}%
    {%
      \glsentrylongpl{##1}##2,
      \space\firstacronymfont{\glsentryshortpl{##1}}\space,
      (\textit{fr : \glsifhasfrlong{##1}%
        {\glsentryfrlong{##1}##2\glsifhasfrshort{##1}{, \space}{}}{}%
        \glsifhasfrshort{##1}{\glsentryfrshort{##1}}{}})%
    }%
    {%
      \glsentrylongpl{##1}##2,
      \space(\firstacronymfont{\glsentryshortpl{##1}})%
    }%
  }%
  % Plural, first letter upper case:
  \renewcommand*{\Genplacrfullformat}[2]{%
    \glsifhasfr{##1}%
    {%
      \Glsentrylongpl{##1}##2,
      \space\firstacronymfont{\glsentryshortpl{##1}}\space,
      (\textit{fr : \glsifhasfrlong{##1}%
        {\glsentryfrlong{##1}##2\glsifhasfrshort{##1}{, \space}{}}{}%
        \glsifhasfrshort{##1}{\glsentryfrshort{##1}}{}})%
    }%
    {%
      \Glsentrylongpl{##1}##2,
      \space(\firstacronymfont{\glsentryshortpl{##1}})%
    }%
  }%
}

\setacronymstyle{fracr}

\newglossarystyle{listfracr}{%
  \setglossarystyle{list}%
  \renewcommand*{\glossentry}[2]{%
  \item[\glsentryitem{##1}%
    \glstarget{##1}{\glossentryname{##1}}]
    \glossentrydesc{##1}%
    \glsifhasfrlong{##1}%
      {\space(\textit{fr : \glsentryfrlong{##1}}\glsifhasfrshort{##1}{,\space}{)}}%
      {\glsifhasfrshort{##1}{\space(\textit{fr : }}{}}%
    \glsifhasfrshort{##1}{\textit{\glsentryfrshort{##1}})}{}%
    \glspostdescription\space ##2}%
}

% -- define fancy quotes
\definecolor{quotemark}{gray}{0.7}
\makeatletter
\def\fquote{%
    \@ifnextchar[{\fquote@i}{\fquote@i[]}%]
           }%
\def\fquote@i[#1]{%
    \def\tempa{#1}%
    \@ifnextchar[{\fquote@ii}{\fquote@ii[]}%]
                 }%
\def\fquote@ii[#1]{%
    \def\tempb{#1}%
    \@ifnextchar[{\fquote@iii}{\fquote@iii[]}%]
                      }%
\def\fquote@iii[#1]{%
    \def\tempc{#1}%
    \vspace{1em}%
    \noindent%
    \begin{list}{}{%
         \setlength{\leftmargin}{0.55\textwidth}%
         \setlength{\rightmargin}{0.01\textwidth}%
                  }%
         \item[]%
         \begin{picture}(0,0)%
         \put(-15,-5){\makebox(0,0){\scalebox{3}{\textcolor{quotemark}{``}}}}%
         \end{picture}%
         \begingroup\itshape}%
 %%%%********************************************************************
 \def\endfquote{%
 \endgroup\par%
 \makebox[0pt][l]{%
 \hspace{0.4\textwidth}%
 \begin{picture}(0,0)(0,0)% 
	 \put(15,15){\makebox(0,0){%
	 \scalebox{3}{\color{quotemark}''}}}%
 \end{picture}}%
 \ifx\tempa\empty%
 \else%
    \ifx\tempc\empty%  
       \hfill\rule{100pt}{0.5pt}\\\mbox{}\hfill\tempa,\ \emph{\tempb}%
   \else%		
       \hfill\rule{100pt}{0.5pt}\\\mbox{}\hfill\tempa,\ \emph{\tempb},\ \tempc%
   \fi\fi\par%
   \vspace{0.5em}%
 \end{list}%
 }%
\makeatother

%% ----------------------------------------------------------------
%% --- change font
%%
%% --- float with adjusting left / right margins using
%% \begin{figure}[t]
%%  \centering
%%  \addtolength{\leftskip}{-3cm}
%%  \addtolength{\rightskip}{-3cm}
%%  % figure content
%% \end{figure}
%%
%% --- custom ("pro") arrays with \toprule, \midrule, \cmidrule and \bottomrule
%% ----------------------------------------------------------------
\renewcommand{\familydefault}{\rmdefault}
\makeatletter
\newcommand*{\centerfloat}{%
  \parindent \z@
  \leftskip \z@ \@plus 1fil \@minus \textwidth
  \rightskip\leftskip
  \parfillskip \z@skip}
\makeatother
