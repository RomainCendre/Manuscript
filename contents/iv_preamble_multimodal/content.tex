\renewcommand{\thechapter}{\roman{chapter}}
\setcounter{chapter}{4}
\setcounter{figure}{0}

\unchapter{Préambule multimodalité}
\label{chap:preamble_multimodal}
Cette troisième et dernière partie va nous permettre de mettre au point et d'évaluer des stratégies permettant la mise en place d'un processus multimodal d'aide au diagnostic du \gls{lm} et \gls{lmm}. Nous présenterons deux des modalités manquantes et reprendrons l'ensemble des éléments en lien avec le \gls{rcm}, vu lors de la \Cref{part:microscopy}. Ce préambule va nous permettre de présenter le jeu de données dans sa globalité et d'amener à nouveau des objectifs de performances en lien avec l'évaluation des experts sur cette base commune.\par


Lors du précédent préambule consacré à la microscopie confocale (\Cref{chap:preamble_microscopy}), nous avons évoqué la recherche de critères clé lors du diagnostic clinique. Ces critères sont présents pour la plupart des modalités d'imagerie.