\renewcommand{\thechapter}{\roman{chapter}}
\setcounter{chapter}{4}
\setcounter{figure}{0}

\unchapter{Conclusion et perspectives}
\label{chap:conclusion}

\section*{État d'avancement}
Le travail présenté tout au long de ce manuscrit a proposé une réponse à la problématique de diagnostic des lésions de la peau en dermatologie, par l'utilisation de la multimodalité. Plus spécifiquement, les méthodes proposées dans ce travail ont été orientées autour de la détection des pathologies de \gls{lm} et de \gls{lmm} dont l'incidence ne cesse de croître au fil des années, et pour lesquelles la réalisation d'un diagnostic clinique par le praticien peut être ardue selon leurs configurations respectives ou stade d'avancement. En s'inspirant notamment des avancées sur les méthodes d'apprentissage profond, ce manuscrit a établi une réflexion sur ce sujet permettant la mise en place d'outils d'aide à la détection précoce de pathologies malignes de la peau.\par

Afin de parvenir à cet objectif, la \Cref{part:contexte} a proposé une présentation des éléments essentiels à la réalisation de ce sujet dont~:
\begin{inlinerate}
    \item une explication clinique \textbf{des principales pathologies de la peau mais surtout du \gls{lm} et de \gls{lmm}},
    \item une présentation de \textbf{l'interaction de la lumière avec la peau et des modalités d'imagerie existantes},
    \item et une description \textbf{des méthodes d'apprentissage sollicitées} dans ce manuscrit pour permettre la mise en place de méthodes de diagnostic.
\end{inlinerate} 
En outre, cette première partie a servi à présenter \textbf{les caractéristiques de la base de données de Cinotti~\al{Cinotti2018}} employée pour évaluer les méthodes d'aide au diagnostic par ordinateur proposées. De même, les apports qui ont été réalisés sur cette base de données telle que la mise à jour des annotations ou encore les modifications d'organisation sont détaillés dans cette même partie.
Par ailleurs, cette présentation de la base de données \textbf{a servi à comprendre les éléments-clés identifiés par les dermatologues sur chacune des modalités} afin de caractériser ces désordres au sein des tissus. La \Cref{part:microscopy} \textbf{a mis en avant les diverses méthodes de diagnostic sur les données de \gls{mcr}} au niveau des images avant de parvenir au diagnostic d'une lésion de cette modalité. Cette étape du manuscrit fut nécessaire à la réalisation du processus multimodal dans son ensemble, cette modalité étant relativement complexe à traiter en l'état. Enfin, la \Cref{part:multimodal} a permis de répondre à la problématique de ce manuscrit en proposant au lecteur diverses méthodes dont le principe permet une prise en charge séquentielle d'une lésion à l'aide des diverses modalités mises à disposition et couramment utilisées en milieu clinique. Les prochains paragraphes commentent les résultats qui ont été obtenus lors de ces deux dernières parties de contribution.\par

Contrairement à des modalités en plein essor telles que la dermatoscopie, la modalité de \gls{mcr} est moins développée et ne propose que très peu de travaux en lien avec les pathologies de \gls{lm} et de \gls{lmm}. Ainsi, la \Cref{part:microscopy} a repris les éléments existants de la littérature dont le contenu sert de base de référence en termes de performances aux méthodes proposées. \textit{D'une part au niveau des images}, le processus de classification mêlant simplicité de mise en œuvre et qualité du diagnostic. Ce processus type a été décrit comme étant composé~:
\begin{inlinerate}
    \item \textbf{d'une extraction de caractéristiques par du transfert de connaissances basé sur l'architecture ResNet-50 pré-entraînée sur ImageNet associé à une couche de Global Pooling - Moyenne},
    \item \textbf{d'une mise à l'échelle des caractéristiques par Minimum / Maximum},
    \item et \textbf{d'un modèle \gls{svm} à noyau linéaire}.
\end{inlinerate} Cette combinaison a permis l'obtention d'une \textbf{valeur de \fscore{} pondérée sur les classes saines, bénignes et malignes de 0,77 associée à un écart-type de 0,04}. Sur la classe maligne, \textbf{une valeur de \fscore{} de 0,82 associée à un écart-type de 0,02 est obtenue, complété par des valeurs de sensibilité de 0,79 et de spécificité de 0,75}. Des améliorations non significatives ont été apportées à l'aide de méthodes exploitant le principe d'échelles multiples ou encore le principe de fenêtre glissante, couplées au précédent processus de classification. De ce fait, ces méthodes n'ont pas été considérées dans la suite de ce manuscrit. \textit{D'autre part au niveau des lésions}, les résultats les plus pertinents ont été obtenus à l'aide d'une extraction de caractéristiques par le transfert de connaissances basé sur l'architecture ResNet-50 pré-entraînée sur ImageNet associée à une couche de Global Pooling - Moyenne et du modèle \gls{nsk} basé sur un principe de \gls{mil}. Une mise à l'échelle étant déjà réalisée par ce modèle, son utilisation en dehors a été rendue caduque. En considérant le diagnostic clinique le plus récent, cette méthode permet d'atteindre \textbf{sur l'ensemble des lésions malignes des valeurs de sensibilité de 0,86 et de spécificité de 0,77}, tandis que \textbf{sur les lésions de type \gls{lm} et \gls{lmm} sont obtenues des valeurs de sensibilité de 0,87 et de spécificité de 0,77}. Ainsi, cette méthode a permis d'obtenir des performances proches de celles obtenues par l'analyse des résultats des experts en \gls{mcr} sur les annotations originales de Cinotti~\al{Cinotti2016}, avec sur l'ensemble des annotations malignes des valeurs de sensibilité de 0,84 et de spécificité de 0,75, tandis que sur les lésions de type \gls{lm} et \gls{lmm} sont obtenues des valeurs de sensibilité de 0,80 et de spécificité de 0,81.\par

Afin de proposer une réponse à la problématique décrite dans ce manuscrit, la \Cref{part:multimodal} a apporté diverses solutions en lien avec la thématique du sujet. Bien que perfectibles, ces méthodes développées sur la modalité de \gls{mcr} permettent la gestion de ces données, qui ont constitué un verrou à la mise en place d'un processus multimodal. Pour les modalités de photographie clinique et de dermatoscopie, des approches par transfert de connaissances ont été préférées aux méthodes par réglage fin représentant un bon compromis entre les performances et ressources matérielles nécessaires. Par la proposition d'un processus sans mémoire, une valeur de \fscore{} pondérée sur les classes bénignes et malignes de 0,80 et associée à un écart-type de 0,10 a été mesurée. Cette valeur a découlé d'une prise en charge~:
\begin{inlinerate}
    \item sur \textbf{la modalité de photographie clinique}~:~d'une lésion maligne considérée comme bénigne,
    \item sur \textbf{la modalité de dermatoscopie}~:~de 10 lésions bénignes dont 4 sont correctement identifiées, de 56 lésions malignes dont 56 lésions sont correctement identifiées,
    \item sur \textbf{la modalité de \gls{mcr}}~:~de 76 lésions bénignes dont 59 lésions sont correctement identifiées, de 81 lésions malignes dont 66 lésions sont correctement identifiées.
\end{inlinerate} Par la proposition d'un processus avec mémoire, une valeur de \fscore{} pondérée sur les classes bénignes et malignes de 0,82 associée à un écart-type de 0,08 a été mesurée. Cette valeur a découlé d'une prise en charge~:
\begin{inlinerate}
    \item sur \textbf{la modalité de photographie clinique}~:~de 2 lésions bénignes correctement identifiées, de 7 lésions malignes dont 4 sont correctement identifiées,
    \item sur \textbf{la modalité de dermatoscopie}~:~de 17 lésions bénignes lésions dont 16 sont correctement identifiées, de 4 lésions malignes dont 2 lésions sont correctement identifiées,
    \item sur \textbf{la modalité de \gls{mcr}}~:~de 67 lésions bénignes dont 41 lésions sont correctement identifiées, de 127 lésions malignes dont 115 lésions sont correctement identifiées.
\end{inlinerate}\par

Pour finir, ces résultats sont bien évidemment perfectibles et ces méthodes ne se substituent pas à l'avis du dermatologue. Bien que de nombreuses dispositions soient prises pour permettre l'obtention de résultats objectifs, les résultats des évaluations sont issus des données de l'étude de Cinotti~\al{Cinotti2016}. Ainsi, ces données sont une fraction infime des situations cliniques et de la diversité de tissus pouvant être rencontrées. La section suivante propose diverses perspectives envisagées afin d'améliorer les résultats actuels.\par
\clearpage

\section*{Perspectives}
Les résultats obtenus et leur analyse ouvrent sur des perspectives intéressantes à notre sens. Tout d'abord, la classification des lésions \gls{mcr} est encore imprécise, et bien que les performances obtenues par réglage fin de \gls{cnn} ne soient pas suffisantes avec la proposition de ce manuscrit, cette solution semble adaptée à ce type de tâche et permet de localiser efficacement des pathologies responsables au sein des images de manière efficiente. En addition, les travaux actuellement proposés et ceux de la littérature ne considèrent à ce jour qu'une détection des tissus pathologiques au sein des images, pourtant les lésions de \gls{lm} et \gls{lmm} se définissent par une prolifération de ces tissus autour des follicules pileux. Une des pistes consiste en une identification des follicules pileux par des techniques de traitement d'images traditionnelles ou par l'utilisation de \gls{cnn} spécialisés dans la détection d'objets, combinée avec l'identification de tissus pathologiques aux alentours de ceux-ci. Une autre limitation de la modalité de \gls{mcr} concerne l'acquisition des données à disposition. Dans l'état actuel, aucune information de localisation spatiale n'est mise à disposition, ainsi une piste possible est de travailler avec les données brutes fournies par l'appareil afin d'accéder à ces données de localisation et envisager des méthodes volumiques par exemple.\par

De plus, de nombreux éléments d'améliorations peuvent être apportés au processus multimodal actuel. Les processus d'extraction de caractéristiques et de prédiction proposés ne sont pas suffisamment spécifiques à la photographique clinique et à la dermatoscopie. Également, certains travaux préconisent un recadrage réalisé manuellement autour des lésions pour la photographie clinique, et non considéré pour la dermatoscopie ; l'utilisation de méthodes automatiques de recadrage peut être envisagée pour améliorer ces performances. De manière similaire, les principes de calibration employés sont à revoir, leur utilisation s'étant révélée inefficace et ayant conduit à une dégradation des résultats. Un autre point d'amélioration du processus multimodal concerne la détermination des seuils de confiance. La proposition actuelle ne parvient à concilier l'optimisation que d'une seule contrainte à savoir d'une métrique au choix de l'utilisateur, et délaisse une seconde contrainte liée à l'optimisation du processus de prise en charge des lésions au fil des modalités. De plus, la recherche des seuils de confiance est réalisée de manière itérative dont l'optimalité peut être contestée.\par

Enfin, les données de ce manuscrit sont délicates à confronter aux résultats issus de l'évaluation des diagnostics des divers experts de l'étude de Cinotti~\al{Cinotti2016}. L'une d'entre elles serait de mettre à jour ces résultats à partir des nouvelles annotations issues de visites de contrôle ultérieures de ces mêmes lésions. Par ailleurs, l'une des contraintes majeures des expérimentations liées aux processus multimodaux est la quantité de lésions mises à disposition. L'une des améliorations à court terme consisterait à travailler en collaboration avec les médecins investigateurs sur l'accroissement de ce nombre de cas dédiés à l'entraînement de ce processus afin d'exclure les hypothèses de dégradation des performances liées à ce nombre de lésions. À plus long terme, il est envisagé d'utiliser des données issues de divers centres cliniques pour vérifier la robustesse des méthodes proposées et exclure une dépendance de celles-ci aux opérateurs et appareillages utilisés.\par

\clearpage

\section*{Valorisations scientifiques}
Pour finir, ces travaux ont donné lieu à de multiples valorisations scientifiques par la publication d'articles de revue, par la participation à des conférences mais également à des groupes de travail. Au moment de l'écriture de ce manuscrit, un article sur la valorisation des travaux de multimodalité présenté dans ce manuscrit est en cours de rédaction. Par ailleurs, un article à orientation médicale sur un projet connexe est prévu sur le thème de la classification de tumeurs des parois buccales et de la gorge par spectrométrie.\par

\textbf{Article de revue~:}
\vspace{-0.1cm}
\begin{itemize}
    \item \bibentry{Cendre2020}
\end{itemize}

\textbf{Conférence internationale~:}
\vspace{-0.2cm}
\begin{itemize}
    \item \bibentry{Cendre2019}
\end{itemize}

\textbf{Conférence nationale~:}
\vspace{-0.2cm}
\begin{itemize}
    \item \bibentry{CendreGretsi}
\end{itemize}

\textbf{Groupe de travail international~:}
\vspace{-0.2cm}
\begin{itemize}
    \item \bibentry{CendreCost}
    \item \bibentry{CendreColourlab}
\end{itemize}

\textbf{Article de revue - En cours~:}
\vspace{-0.1cm}
\begin{itemize}
    \item \bibentry{CendreMultimodal}
    \item \bibentry{CendreMultispectral}
\end{itemize}

\textbf{Codes source~:}
\vspace{-0.2cm}
\begin{itemize}
    \item \href{https://github.com/RomainCendre/classification}{https://github.com/RomainCendre/classification}
\end{itemize}